% 2014
% Maciej Szeptuch
% IIUWr

\documentclass[12pt,leqno]{article}

\usepackage[utf8]{inputenc}
\usepackage{polski}
\usepackage{a4wide}
\usepackage[cm]{fullpage}

\usepackage{graphicx}
\usepackage{epstopdf}
\usepackage{amsmath,amssymb}
\usepackage{bm}
\usepackage{amsthm}

%% Kropka po numerze paragrafu, podparagrafu, itp.
\makeatletter
    \renewcommand\@seccntformat[1]{\csname the#1\endcsname.\quad}
    \renewcommand\numberline[1]{#1.\hskip0.7em}
\makeatother

%%%%%%%%%%%%%%%%%%%%%%%%%%%%%%%%%%%%%%%%%%%%%%%%%%%%%%%%%%%%%%%%%%%%%%%%%%%%%%%%

\title{\LARGE \textbf{{Systemy Operacyjne}}\\
      {\Large Pracownia -- Zadanie 1 -- Semafory}\\
}
\author{Maciej Szeptuch}
\date{Wrocław, \today}

\begin{document}
\thispagestyle{empty}
\maketitle

Pierwsze, obowiązkowe zadanie na pracownie z systemów operacyjnych było związane
z semaforami i ich użyciem np.\ do rozwiązania pewnego problemu synchronizacyjnego.
Jako że to zadanie nie było specjalnie skomplikowane, ja postanowiłem spróbować
swoich sił z implementacją również samych semaforów. Istnieje wiele różnych podejść
do tego problemu, ja postanowiłem wykorzystać operacje atomową porównaj-i-zamień
(\textbf{Compare-and-swap, CAS}), poprzez dostępną w kompilatorze gcc funkcje
wbudowaną \textit{\_\_sync\_bool\_compare\_and\_swap}. Korzystając z tej operacji
implementacja semaforów jest dosyć oczywista, bardzo krótki i zwięzły kod źródłowy
(\textit{semaphore\_spin.h}) został załączony do tego dokumentu w folderze
\textit{semaphore}, wraz z przykładami użycia oraz rozwiązaniem problemu
producenta-konsumenta. \\
Ta bardzo prosta implementacja ma jedną wadę, działa na zasadzie aktywnego oczekiwania
tzw.\ wirująca blokada(\textbf{spinlock}), mianowicie oczekiwanie na zwolnienie
blokady polega na ciągłym badaniu jej stanu. W związku z tym zaimplementowałem również
rozszerzenie tych semaforów o listę oczekujących procesów, której działanie polega
na tym, że gdy proces/wątek wyrazi chęć dostępu do zablokowanego zasobu zostanie
on uśpiony, dołączony do listy i w momencie, w którym zasób zostanie odblokowany
pierwszy proces z listy dostaje sygnał obudzenia i tak aż skończą się żądania.
W celu zapewnienia spójności listy pomiędzy procesami/wątkami również należy
wykorzystać pewne sposoby synchronizacji -- tylko jeden wątek może mieć w danym
momencie dostęp do listy. Aby to zapewnić została użyta wcześniejsza implementacja
semaforów, operacje na liście są bardzo szybkie w związku z czym nie jest dla nas
problemem aktywne oczekiwanie. Implementacja jest dostępna w postaci pliku
\textit{semaphore\_list.h} w załączonym katalogu \textit{semaphore} z przykładami
tak jak poprzednia. \\
\\
Aby uruchomić załączone przykłady należy je najpierw skompilować. W tym celu
wystarczy w katalogu semaphore wywołać polecenie \textit{make}. Po czym możemy
uruchomić program \textit{simple}, które przyjmuje jako argumenty liczbę wątków dla testu
zwykłych semaforów oraz liczbę wątków dla testu semaforów opartych na listach,
po czym wykonuje on bardzo proste testy polegające na uruchomieniu podanej liczby wątków
i kolejnym blokowaniu zasobu przez następne wątki. \\
Dostępne są także dwa programy \textit{producer-consumer} oraz
\textit{producer-consumer-sys} są to implementacje rozwiązania problemu
producenta-konsumenta z wykorzystaniem odpowiednio mojej implementacji semaforów oraz
systemowej. Za argumenty przyjmują one liczbę odpowiednio wątków producentów oraz
konsumentów.
\end{document}
