% 2014
% Maciej Szeptuch
% IIUWr

\documentclass[12pt,leqno]{article}

\usepackage[utf8]{inputenc}
\usepackage{polski}
\usepackage{a4wide}
\usepackage[cm]{fullpage}

\usepackage{hyperref}
\usepackage{graphicx}
\usepackage{epstopdf}
\usepackage{amsmath,amssymb}
\usepackage{bm}
\usepackage{amsthm}

%% Kropka po numerze paragrafu, podparagrafu, itp.
\makeatletter
    \renewcommand\@seccntformat[1]{\csname the#1\endcsname.\quad}
    \renewcommand\numberline[1]{#1.\hskip0.7em}
\makeatother

%%%%%%%%%%%%%%%%%%%%%%%%%%%%%%%%%%%%%%%%%%%%%%%%%%%%%%%%%%%%%%%%%%%%%%%%%%%%%%%%

\title{\LARGE \textbf{{Systemy Operacyjne}}\\
      {\Large Pracownia -- Zadanie 3 -- Skrypt powłoki -- Oszczędzanie energii}\\
}
\author{Maciej Szeptuch}
\date{Wrocław, \today}

\begin{document}
\thispagestyle{empty}
\maketitle

Trzecim zadaniem jest skrypt \textit{pwrsaver} napisany w języku powłoki bash,
służący w pewnym sensie do zarządzania oszczędzaniem energii. Napisałem go
w celu maksymalizacji czasu działania na baterii mojego laptopa. Jego działanie
sprowadza się do włączania i wyłączania odpowiednich rzadko używanych usług
systemowych, lub w tym momencie zwyczajnie niepotrzebnych. Oraz takim samym
działaniem w stosunku do załadowanych modułów w jądrze Linux. \\
Skrypt pozwala m.in.\ na włączanie/wyłącznie podsystemu dźwięku, obsługi sieci
(bezprzewodowej, przewodowej, GSM), bluetooth, czytnika kart SD, całego
podsystemu USB, czytnika kart inteligentnych i wielu innych. Swego czasu z jego
pomocą można było przedłużyć czas działania na baterii o 1.5-2 godziny. Co prawda
niewiele można robić gdy chce się osiągnąć taki zysk (system działał tylko w konsoli)
ale w zupełności wystarczało to np.\ do pisania kodu, czytania dokumentów tekstowych,
dokumentacji. \\
\\
Jego użycie sprowadza się do wywołania z odpowiednimi argumentami, które wyświetlają
się w momencie gdy spróbujemy go uruchomić bez żadnych. Np.\ \textit{./pwrsaver r sound stop}
spowoduje wyłączenie podsystemu dźwięku i wszystkiego co z nim związane. \\
Z ważnych informacji: skrypt do działania wymaga uprawnień administratora, które można
uzyskać korzystając np.\ z \textit{sudo} czy też \textit{su}.
\end{document}
