% 2014
% Maciej Szeptuch
% IIUWr

\documentclass[12pt,leqno]{article}

\usepackage[utf8]{inputenc}
\usepackage{polski}
\usepackage{a4wide}
\usepackage[cm]{fullpage}

\usepackage{hyperref}
\usepackage{graphicx}
\usepackage{epstopdf}
\usepackage{amsmath,amssymb}
\usepackage{bm}
\usepackage{amsthm}

%% Kropka po numerze paragrafu, podparagrafu, itp.
\makeatletter
    \renewcommand\@seccntformat[1]{\csname the#1\endcsname.\quad}
    \renewcommand\numberline[1]{#1.\hskip0.7em}
\makeatother

%%%%%%%%%%%%%%%%%%%%%%%%%%%%%%%%%%%%%%%%%%%%%%%%%%%%%%%%%%%%%%%%%%%%%%%%%%%%%%%%

\title{\LARGE \textbf{{Systemy Operacyjne}}\\
      {\Large Pracownia -- Zadanie 2 -- Moduł do jądra Linuksa}\\
}
\author{Maciej Szeptuch}
\date{Wrocław, \today}

\begin{document}
\thispagestyle{empty}
\maketitle

Jako drugie zadanie postanowiłem spróbować swoich sił z napisaniem prostego modułu
do jądra systemu Linux. Zaimplementowałem proste wirtualne urządzenie znakowe,
które udostępnia nieskończony ciąg znaków w postaci "Hello World!" występujących
jeden za drugim. Zadanie również okazało się niezbyt skomplikowane. Korzystając
z ogólnodostępnych zasobów np.\ \url{http://www.tldp.org/LDP/lkmpg/2.6/html/lkmpg.html}
można krok po kroku nauczyć się pisać takie moduły w przeciągu kilkudziesięciu minut. \\
\\
Moduł jest dostępny w katalogu \textit{module}. Aby go użyć należy go najpierw skompilować.
W tym celu wystarczy przejść do wspomnianego folderu i wydać polecenie \textit{make} po
czym w bieżącym katalogu powinien pojawić się moduł \textit{hello.ko}. Załadować go można
dowolnym dostępnym poleceniem np.\ \textit{insmod hello.ko}. Po załadowaniu w katalogu
\textit{/dev} powinno pojawić się urządzenie o jakże skomplikowanej nazwie \textbf{hello}.
W momencie w którym będziemy próbowali z niego czytać np.\ używając programu \textit{cat}
na ekranie powinny się pojawiać "Hello World!". Zapis do urządzenia jest niedostępny
w związku z czym w momencie próby zapisu czegokolwiek powinien pojawiać się błąd. \\
Po skończonym testowaniu można usunąć moduł z jądra korzystając np.\ z polecenia
\textit{rmmod}. \\
Z ciekawych informacji, w momencie ładowania i usuwania modułu z jądra w dziennikach
systemowych powinny pojawiać się informacje z samego modułu w postaci "Hello World! (un)loaded".
\end{document}
